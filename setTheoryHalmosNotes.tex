\documentclass{article}
\usepackage{amsmath}
\usepackage{amsfonts}
\usepackage{amssymb}
\usepackage{faktor}
\usepackage[mathscr]{euscript}

\title{Notes on Naive Set Theory by Halmos}
\author{Benjamin Basseri}
\date{ }

\begin{document}

\maketitle

\section{Axiom of Extension}

Sets are equal if and only if they contain exactly the same members.

\section{Axiom of Specification}

\subsection{The Axiom of Specification} Sets can be filtered by logical conditions to determine subsets.

\subsection{Russell's Paradox} There cannot be a `universal' set that contains everything. 

Let $B = \{x \in A: x \not\in x\}$; the elements of $A$ that don't contain themselves. While $B$ is a perfectly valid set, there is no set $A$ that it can belong to.

\textbf{Proof} Assume toward a contradiction that $B \in A$ for some set $A$. We cannot have $B \in B$ since that contradicts the specification for $B$ membership. But if $B \not\in B$ then $B$ qualifies for $B$ membership, which means it \textit{is} in $B$. Since this is a contradiction, there cannot be a set $A$ that has $B$ as a member.

\section{Unordered Pairs}

\subsection{The Empty Set} The empty set $\varnothing$ has no members. It's considered a subset of every set since, for any set $A$, $\varnothing$ has no members that are not in $A$.

\subsection{The Axiom of Pairing} For any two sets there is a set they both belong to. 

\subsection{Unordered Pairs} We can use specification to make an unordered pair $\{a, b\}$ out of any distinct $a$ and $b$, a set that contains them and nothing else. 

\subsection{Specifying sets with only a proposition} There are some special sets that can be specified by only a proposition $S(x)$, such as $\{x: x \neq x\} = \varnothing$ or $\{x: x = a\} = \{a\}$. But in general, the axiom of specification requires a proposition $S(x)$ be joined with a set $A$ as in $\{x \in A: S(x)\}$.

\section{Unions and Intersections}

\subsection{Axiom of unions} A set can be formed from the members of each set in a collection of sets.

\subsection{Union and Intersection are associative and commutative} Unions and intersections can be thought of as specific logical propositions involving \textit{or} and \textit{and}. Since logical \textit{or} and logical \textit{and} are associative and commutative in Boolean algebra, unions and intersections are as well.

\subsection{Intersections with the empty set} For a collection of sets $\mathcal{C}$ define 
$$\bigcup \mathcal{C} = \{x: x \in X \text{ and } X \in \mathcal{C}\}$$

If $\mathcal{C}$ is empty, then no $x$ can satisfy the specification and the intersection is empty. Halmos makes a technical point about requiring $\mathcal{C}$ to be non-empty. But this is an artifact of the way he phrased intersecting a collection, in order to avoid $x$'s vacuously satisfying the intersection condition. This is not needed if the specification above is used.

\section{Complements and Powers}

\subsection{Complements} The set difference $A - B$ are the elements of $A$ that are not also in $B$. This is the \textit{relative complement} of $B$ in $A$. We can write this as $B'$ when the set $A$ is clear from context.

\subsection{Axiom of powers} For each set $A$ there is a set $\mathcal{P}(A)$ whose members are all possible subsets of $A$. 

\section{Ordered Pairs}

\subsection{Ordered Pairs from Sets} To encode an ordering of elements, make a collection where each element to be ordered is replaced by a set containing that element and all its predecessors.
$$X = \{a, b, c\}, \text{ order desired: } (c, b, a)$$
$$(c, b, a) \longmapsto \{\{c\}, \{c, b\}, \{c, b, a\}\}$$

\subsection{Equality of Ordered Pairs} With ordered pairs if $(a, b) = (x, y)$ then each coordinate must be equal: $a = x, b = y$. By the encoding above, each ordered pair has exactly one singleton and one unordered pair, so there is no ambiguity which is which.

\section{Relations}

Using sets, we can encode relations as ordered pairs: $(x, y)$. A set of ordered pairs can then collect all the elements $x, y$ such that $x$ has the given relation to $y$.

\section{Functions}

\subsection{Defining functions} 

On $X \times Y$, a function pairs every element of $X$ with exactly one element of $Y$. Compare to a relation, which does not require every $x \in X$ to be in a pair, and doesn't require a unique $y$ to be paired with any $x$.

\subsection{The Canonical Map} The canonical map is the function that maps $x \in X$ to its equivalence class in $\faktor{X}{R}$.

\subsection{Characteristic Function} For $A \subset X$, the characteristic function $\chi_A(x)$ = 1 if $x \in A$ and 0 if $x \in X - A$. Note that $A$ is a subset of $X$, so $A \in \mathscr{P}(X)$, but $\chi_A$ is a function, so it is a member of $2^X$. However, we can naturally associate a subset to its characteristic function with the correspondence $A \longrightarrow \chi_A$.

\section{Families}

\subsection{Index sets and families} It may be convenient to give certain elements of $X$ a label such as $x_i$ or $x_\alpha$. To do so we establish a function $I \longrightarrow X$ from the \textit{index set} $I$ (the set of labels) to $X$. Like any function this pairs an $i \in I$ with a specific $x \in X$. Thinking of $X$ being \textit{indexed} by $I$, we call this function the \textit{family} $\{x_i\}$.

\subsection{Unioning a family} For a family $\{A_i\}$, the notation $\bigcup_{i \in I} A_i$ or $\bigcup_i A_i$ indicates the union of the range of the family


\end{document}